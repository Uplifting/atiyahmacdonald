\documentclass[book,12pt,oneside,openany]{memoir}
\usepackage[utf8x]{inputenc}
\usepackage[english]{babel}
\usepackage{amsmath}
\usepackage{amssymb}
\usepackage{url}

% for placeholder text
\usepackage{lipsum}

\title{Solutions to \emph{Introduction to Commutative Algebra} by Atiyah and Macdonald}
\author{Aditya Gudibanda}
\begin{document}

\maketitle

\chapter{}

\section{}
\subsection{Problem}
Let $x$ be a nilpotent element of a ring $A$. Show that $1 + x$ is a unit of  $A$. Deduce that the sum of a nilpotent element and a unit is a unit.

\subsection{Solution}
Note the identity $\left(\sum_{i = 0}^k x_i \right) (1-x) =  1- x^{k+1}$. Since $x$ is nilpotent,  $x^l = 0$ for some $l \geq 1$. Thus, if we set $k = l-1$ in the identity above, we see that $1-x$ has a multiplicative inverse, and is therefore  a unit. 

To prove the original statement, given that $x$ is a nilpotent unit, it is clear that $-x$ is nilpotent as well, and by the logic above, $1 - (-x) = 1 + x$ is a unit, as desired.

Suppose $u$ is an arbitrary unit. Since $x$ is nilpotent, $u^{-1}x$ is nilpotent as well, so by the above, $u^{-1}x + 1$ is a unit. Since the product of units is a unit, $u(u^{-1}x + 1) = x + u$ is a unit, as desired.

\section{}
\subsection{Problem}
Let $A$ be a ring and let $A[x]$ be the ring of polynomials in an indeterminate $x$, with coefficients in $A$. Let $f = a_0 + a_1x + \cdots + a_n x^n \in A[x]$. Prove that

i) $f$ is a unit in $A[x] \Leftrightarrow a_0$ is a unit in $A$ and $a_1, \ldots, a_n$ are nilpotent. 

ii) $f$ is nilpotent $\Leftrightarrow a_0, a_1, \ldots, a_n$ are nilpotent.

iii) $f$ is a zero-divisor $\Leftrightarrow$ there exists $a \neq 0$ in $A$ such that $af = 0$.

iv) $f$ is said to be primitive if $(a_0, a_1, \ldots, a_n) = (1)$. Prove that if $f,g \in A[x]$, then $fg$ is primitive $\Leftrightarrow f$ and $g$ are primitive.

\subsection{Solution}

i) 


\section{}
\subsection{Problem}
Generalize the results of Exercise 2 to a polynomial ring $A[x_1, \ldots, x_r]$ in several indeterminates.

\subsection{Solution}

\section{}
\subsection{Problem}
In the ring $A[x]$, the Jacobson radical is equal to the nilradical.

\subsection{Solution}


\section{}
\subsection{Problem}
Let $A$ be a ring and let $A[[x]]$ be the ring of formal power series $f = \sum_{n=0}^{\infty} a_n x^n$ with coefficients in $A$. Show that 

i) $f$ is a unit in $A[[x]] \Leftrightarrow a_0$ is a unit in $A$.

ii) If $f$ is nilpotent, then $a_n$ is nilpotent for all $n \geq 0$. Is the converse true?

iii) $f$ belongs to the Jacobson radical of $A[[x]] \Leftrightarrow a_0$ belongs to the Jacobson radical of $A$.

iv) The contraction of a maximal ideal $\mathfrak{m}$ of $A[[x]]$ is a maximal ideal of $A$, and $\mathfrak{m}$ is generated by $\mathfrak{m}^c$ and $x$.

v) Every prime ideal of $A$ is the contraction of a prime ideal of $A[[x]]$.


\subsection{Solution}

\section{}
\subsection{Problem}
A ring $A$ is such that every ideal not contained in the nilradical contains a nonzero idempotent (that is, an element $e$ such that $e^2 = e \neq 0$). Prove that the nilradical and Jacobson radical of $A$ are equal.
\subsection{Solution}



\section{}
\subsection{Problem}
Let $A$ be a ring in which every element $x$ satisfies $x^n = x$ for some $n > 1$ (depending on $x$). Show that every prime ideal in $A$ is maximal.
\subsection{Solution}
Let $\mathfrak{a}$ be a prime ideal. Then $A/\mathfrak{a}$ is an integral domain. Suppose that $x \in A$. Then $x^n = x$ for some $n > 1$. Let $\bar{x}$ be the image of $x$ under the projection to the quotient by $\mathfrak{a}$. Then $\bar{x}^n - \bar{x} = \bar{x} (\bar{x}^{n-1} - 1) = 0$. Since $A/\mathfrak{a}$ is an integral domain, this implies that either $\bar{x} = 0$ or $\bar{x}^{n-1} = 1$. If $\bar{x}^{n-1} = 1$, then $\bar{x} \cdot \bar{x}^{n-2} = 1$, so $\bar{x}$ has an inverse. This implies that for every $\bar{x} \in A/\mathfrak{a}$, either $\bar{x} = 0$ or $\bar{x}$ has an inverse. Thus, $A/\mathfrak{a}$ is a field, so $\mathfrak{a}$ is maximal, as desired.


\section{}
\subsection{Problem}
Let $A$ be a ring $\neq 0$. Show that the set of prime ideals of $A$ has minimal elements with respect to inclusion.
\subsection{Solution}



\section{}
\subsection{Problem}
Let $\mathfrak{a}$ be an ideal $\neq (1)$ in a ring $A$. Show that $\mathfrak{a} = r(\mathfrak{a}) \Leftrightarrow \mathfrak{a}$ is an intersection of prime ideals.
\subsection{Solution}

$\Leftarrow$: Suppose $\mathfrak{a}$ is an intersection of prime ideals. Clearly, $\mathfrak{a} \subset r(\mathfrak{a})$. Suppose $x \notin \mathfrak{a}$. Then there is some prime ideal $\mathfrak{p}$ which $x$ is not in, but in which $\mathfrak{a}$ is contained. Thus, no power of $x$ will ever be in $\mathfrak{p}$, so no power of $x$ will ever be in $\mathfrak{a}$. Thus, $\mathfrak{a} = r(\mathfrak{a})$, as desired.

$\Rightarrow$:  Simply apply Proposition 1.14 - the radical of an ideal $\mathfrak{a}$, in this case $\mathfrak{a}$, is the intersection of the prime ideals which contain $\mathfrak{a}$.
\section{}
\subsection{Problem}
Let $A$ be a ring, $\mathcal{R}$ its nilradical. Show that the following are equivalent:

i) $A$ has exactly one prime ideal

ii) every element of $A$ is either a unit or nilpotent

iii) $A/\mathcal{R}$ is a field
\subsection{Solution}



\section{}
\subsection{Problem}
A ring $A$ is Boolean if $x^2 = x$ for all $x \in A$. In a Boolean ring $A$, show that 

i) $2x = 0$ for all $x \in A$

ii) every prime ideal $\mathfrak{p}$ is maximal, and $A/\mathfrak{p}$ is a field with two elements

iii) every finitely generated ideal in $A$ is principal.
\subsection{Solution}

.

i) $(x +1)^2 = x^2 + 2x + 1 = x + 2x + 1 = x + 1 \Rightarrow 2x = 0$, as desired.

ii) Let $\mathfrak{p}$ be a prime ideal. Then every element in $A/\mathfrak{p}$ satisfies $\bar{x}^2 = \bar{x}$. Since $A/\mathfrak{p}$ is an integral domain, this means that $\bar{x} = 0$ or $\bar{x} = 1$. Thus, $A/\mathfrak{p}$ has only two elements and is hence a field, so $\mathfrak{p}$ is maximal, as desired.

iii) It suffices to be able to reduce the ideal $(a, b)$ and show that it is equivalent to another ideal $(c)$. To this end, let $c = ab + a + b$. It is clear that $c \in (a,b)$. Next, $ca = a^2b + a^2 + ab = ab + a + ab = a$. Next, $cb = ab^2 + ab + b^2 = ab + ab + b = b$. Thus, $(a,b) = (c)$. So given any finite set of generators, we can reduce these to one generator, by induction, by reducing the number of generators one by one.


\section{}
\subsection{Problem}
A local ring contains no idempotent $\neq 0,1$.
\subsection{Solution}
Let $x \in A$, the local ring. Let $\mathfrak{m}$ be the maximal ideal of $A$. Suppose $x$ is idempotent. Then in $A/\mathfrak{m}$, $\bar{x}^2 = \bar{x} \Rightarrow \bar{x} ( \bar{x} - 1) = 0$. Since $A/\mathfrak{m}$ is a field, this implies that either $\bar{x} = 0$ or $\bar{x} = 1$.

If $\bar{x} = 1$, then $x$ is a unit, so if $x^2 = x$, then $x(x - 1) = 0$, so $x$ is either $0$ or $1$. $0$ is not a unit, so $x = 1$.

If $\bar{x} = 0$, then $x \in \mathfrak{m}$, so $x - 1$ is a unit. Since $x(x - 1) = 0$, this means that $x = 0$.

Thus, $x$ is either $0$ or $1$, as desired.


\section{}
\subsection{Problem}
Let $K$ be a field and let $\Sigma$ be the set of all irreducible monic polynomials $f$ in one indeterminate with coefficients in $K$. Let $A$ be the polynomial ring over $K$ generated by indeterminates $x_f$, one for each $f \in \Sigma$. Let $\mathfrak{a}$ be the ideal of $A$ generated by the polynomials $f(x_f)$ for all $f \in \Sigma$. Show that $\mathfrak{a} \neq (1)$.

Let $\mathfrak{m}$ be the maximal ideal of $A$ containing $\mathfrak{a}$, and let $K_1 = A/\mathfrak{m}$. Then $K_1$ is an extension field of $K$ in which each $f \in \Sigma$ has a root. Repeat the construction with $K_1$ in place of K, obtaining a field $K_2$, and so on. Let $L = \cup_{n=1}^{\infty} K_n$. Then $L$ is a field in which each $f \in \Sigma$ splits completely into linear factors. Let $\bar{K}$ be the set of all elements of $L$ which are algebraic over $K$. Then $\bar{K}$ is an algebraic closure of $K$.
\subsection{Solution}



\section{}
\subsection{Problem}
In a ring $A$, let $\Sigma$ be the set of all ideals in which every element is a zero-divisor. Show that the set $\Sigma$ has maximal elements and that every maximal element of $\Sigma$ is a prime ideal. Hence the set of zero-divisors of $A$ is a union of prime ideals.
\subsection{Solution}



\section{}
\subsection{Problem}
Let $A$ be a ring and let $X$ be the set of all prime ideals of $A$. For each subset $E$ of $A$, let $V(E)$ denote the set of all prime ideals of $A$ which contain $E$. Prove that

i) if $\mathfrak{a}$ is the ideal generated by $E$, then $V(E) = V(\mathfrak{a}) = V(r(\mathfrak{a}))$.

ii) $V(0) = X, V(1) = \emptyset$.

iii) If $(E_i)_{i \in I}$ is any family of subsets of $A$, then \[V (\cup_{i \in I} E_i ) = \cap_{i \in I} V(E_i).\]

iv) $V(\mathfrak{a} \cap \mathfrak{b}) = V(\mathfrak{a}\mathfrak{b}) = V(\mathfrak{a}) \cup V(\mathfrak{b})$ for any ideals $\mathfrak{a}, \mathfrak{b}$ of $A$.

\subsection{Solution}



\section{}
\subsection{Problem}
Draw pictures of Spec $\mathbb{Z}$, Spec $\mathbb{R}$, Spec $\mathbb{C}[x]$, Spec $\mathbb{R}[x]$, Spec $\mathbb{Z}[x]$.
\subsection{Solution}



\section{}
\subsection{Problem}

\subsection{Solution}



\section{}
\subsection{Problem}

\subsection{Solution}



\section{}
\subsection{Problem}
A topological space $X$ is said to be irreducible if $X \neq \emptyset$ and if every pair of non-empty open sets in $X$ intersect, or equivalently if every non-empty open set is dense in $X$. Show that Spec $A$ is irreducible if and only if the nilradical of $A$ is a prime ideal.
\subsection{Solution}



\section{}
\subsection{Problem}

\subsection{Solution}



\chapter{}

\section{}
\subsection{Problem}
Show that $\left( \mathbb{Z}/m\mathbb{Z} \right) \otimes_{\mathbb{Z}}  \left( \mathbb{Z}/n\mathbb{Z} \right) = 0$ if $m,n$ are coprime.
\subsection{Solution}
The tensor product is generated by all possible tensor products of pairs of generators from the two rings we are tensoring. Both rings are cyclic and are generated by 1. So the tensor product is generated by $1 \otimes 1$.  Now, note that $m (1 \otimes 1) = m \otimes 1 = 0 \otimes 1 = 0$, and $n (1 \otimes 1) = 1 \otimes n = 1 \otimes 0 = 0$. Thus, the order of $1 \otimes 1$ must divide both $m$ and $n$. However, since $m$ and $n$ are coprime, the only such positive number is 1. Thus, $1 \otimes 1$ has order 1, so $\left( \mathbb{Z}/m\mathbb{Z} \right) \otimes_{\mathbb{Z}}  \left( \mathbb{Z}/n\mathbb{Z} \right)$ is a ring that contains only one element, which is therefore 0, as desired.

\section{}
\subsection{Problem}
Let $A$ be a ring, $\mathfrak{a}$ an ideal, $M$ an $A$-module. Show that $(A/\mathfrak{a}) \otimes_{A} M$ is isomorphic to $M/\mathfrak{a}M$.

\subsection{Solution}
Consider the exact sequence $0 \rightarrow \mathfrak{a} \rightarrow A \rightarrow A/\mathfrak{a} \rightarrow 0$. We now tensor this with $M$ to get $0 \rightarrow \mathfrak{a} \otimes_{A} M \rightarrow A \otimes_{A} M \rightarrow A/\mathfrak{a} \otimes_{A} M \rightarrow 0$. By the first isomorphism theorem, we therefore have that $A/\mathfrak{a} \otimes_{A} M \cong A \otimes_{A} M / \mathfrak{a} \otimes_A M$. Clearly, $A \otimes_A M \cong M$ and $\mathfrak{a} \otimes_{A} M \cong \mathfrak{a} M$, so we have $A/\mathfrak{a} \otimes_{A} M \cong M/\mathfrak{a}M$, as desired.

\section{}

\subsection{Problem}
Let $A$ be a local ring, $M$ and $N$ finitely generated $A$-modules. Prove that if $M \otimes N = 0$, then $M = 0$ or $N = 0$.

\subsection{Solution}
Let $\mathfrak{m}$ be the maximal ideal of $A$ and let $k = A/\mathfrak{m}$ be the residue field. Then $M_k = k \otimes_A M \cong M/\mathfrak{m}M$ by 3.2. Since $A$ is a local ring, $\mathfrak{m}$ is the only maximal ideal of $A$, so by Nakayama's lemma, if $M_k = 0$, then $M = 0$. 

Since $M \otimes_{A} N = 0$, $(M \otimes_{A} N)_k = M_k \otimes_{A} N_k = 0$. Since $M_k$ and $N_k$ are both vector spaces over a field, this implies that either $M_k = 0$ or $N_k = 0$, which implies that either $M = 0$ or $N= 0$, as desired.

\section{}
\subsection{Problem}
Let $M_i (i \in I)$ be any family of $A$-modules, and let $M$ be their direct sum. Prove that $M$ is flat $\Leftrightarrow$ each $M_i$ is flat.

\subsection{Solution}
$\Rightarrow:$ Suppose that $M$ is flat. The maximal ideals of $M$ are $\{\times_{i \neq j} M_i  \times \mathfrak{m}_j | j \in I, \mathfrak{m_j} \in M_j\}$, where $\mathfrak{m}_j$ are the maximal ideals of $M_j$. The quotients of all these maximal ideals must also be flat. These are all $M_j/\mathfrak{m}_j$, over all $j \in I$ and over all $\mathfrak{m}_j$ which are maximal in $M_j$. For a specific $j$, this implies that all the maximal ideals are flat. This implies that $M_j$ is flat. So this direction is done.

$\Leftarrow:$ Just go the other direction.


\section{}
\subsection{Problem}
Let $A[x]$ be the ring of polynomials in one indeterminate over a ring $A$. Prove that $A[x]$ is a flat $A$-algebra.
\subsection{Solution}



\section{}
\subsection{Problem}
For any $A$-module, let $M[x]$ denote the set of all polynomials in $x$ with coefficients in $M$, that is to say expressions of the form \[m_0 + m_1x + \cdots + m_rx^r (m_t \in M)\]

Defining the product of an element of $A[x]$ and an element of $M[x]$ in the obvious way, show that $M[x] \cong A[x] \otimes_{A} M$.

\subsection{Solution}
From Proposition 2.14, we know that $(M + N) \otimes P \cong (M \otimes P) + (N \otimes P)$. We can consider $M[x]$ to be an infinite direct sum of $M$, and $A[x]$ to be an infinite direct sum of $A$. From Proposition 2.14, we know that when $M$ is a module of $A$, $A \otimes M \cong M$. These two facts immediately imply the desired conclusion, $M[x] \cong A[x] \otimes_A M$.


\section{}
\subsection{Problem}
Let $\mathfrak{p}$ be a prime ideal of $A$. Show that $\mathfrak{p}[x]$ is a prime ideal in $A[x]$.

\subsection{Solution}
Let $f,g \in A[x] - \mathfrak{p}[x]$. Let \[f = a_0 + a_1x + a_2x^2 + \cdots a_n x^n\] and let \[g = b_0 + b_1 x + b_2 x^2 + \cdots + b_m x^m\]

Suppose $a_i$ is the coefficient with smallest index that is not in $\mathfrak{p}$, and let $b_j$ be defined likewise. These must both exist because $f,g \notin \mathfrak{p}[x]$. Then consider the coefficient of $x^{i+j}$ in $fg$. WLOG, let $i < j$. Then the coefficient of $x^{i+j}$ in $fg$ is \[\sum_{r = 0}^{i-1} a_r b_{i+j-r} + a_i b_j + \sum_{s = 0}^{j-1} a_{i+j-s}b_s\]

Both the sums are in $\mathfrak{p}$ because $a_r \in \mathfrak{p}$ for $r < i$ and $b_s \in \mathfrak{p}$ for $s < j$, by the definitions of $i$ and $j$. But $a_i$ and $b_j$ are not in $\mathfrak{p}$, and $\mathfrak{p}$ is prime, so $a_i b_j$ is not in $\mathfrak{p}$. Thus, $fg$ is not in $\mathfrak{p}[x]$. Thus, $\mathfrak{p}[x]$ is prime in $A[x]$, as desired.

\section{}
\subsection{Problem}

i) If $M$ and $N$ are flat $A$-modules, then so is $M \otimes_{A} N $.

ii) If $B$ is a flat $A$-algebra and $N$ is a flat $B$-module, then $N$ is flat as an $A$-module.

\subsection{Solution}

i) Let $0 \rightarrow B \rightarrow C \rightarrow D \rightarrow 0$ be an exact sequence of $A$-modules. Since $M$ is flat, \[0 \rightarrow B \otimes_A M \rightarrow C \otimes_A M \rightarrow D \otimes_A M \rightarrow 0\] is exact. Since $N$ is flat, 
\[0 \rightarrow (B \otimes_A M) \otimes_A N \rightarrow (C \otimes_A M) \otimes_A N \rightarrow (D \otimes_A M) \otimes_A N \rightarrow 0\] is also exact. By Proposition 2.14, $(M \otimes N) \otimes P \cong M \otimes (N \otimes P)$, so therefore we have 
\[0 \rightarrow B \otimes_A (M \otimes_A N) \rightarrow C \otimes_A (M \otimes_A N) \rightarrow D \otimes_A (M \otimes_A N) \rightarrow 0\] is exact. Thus, $M \otimes_A N$ is flat, as desired.

ii) Let $0 \rightarrow C \rightarrow D \rightarrow E \rightarrow 0$ be an exact sequence of $A$-modules. Since $B$ is a flat $A$-algebra, \[0 \rightarrow C \otimes_A B \rightarrow D \otimes_A B \rightarrow E \otimes_A B \rightarrow 0\] is exact. Since $N$ is a flat $B$-module, \[0 \rightarrow C \otimes_A B \otimes_B N \rightarrow D \otimes_A B \otimes_B N \rightarrow E \otimes_A B \otimes_B N \rightarrow 0\] 

But $B \otimes_B N \cong N$, so we get 
 \[0 \rightarrow C \otimes_A  N \rightarrow D \otimes_A N \rightarrow E \otimes_A  N \rightarrow 0\] 

which implies that $N$ is flat as an $A$-module, as desired.
\section{}
\subsection{Problem}
Let $0 \rightarrow M' \rightarrow M \rightarrow M'' \rightarrow 0$ be an exact sequence of $A$-modules. If $M'$ and $M''$ are finitely generated, then so is $M$.

\subsection{Solution}
Let $u_1, u_2, \cdots, u_m$ be the generators of $M'$ and let $v_1, v_2, \cdots v_n$ be the generators of $M''$. Let $w_i = f(u_i), 1 \leq i \leq m$. Since $M$ surjects onto $M''$, for each of the $v_i$, there is an $x_i$ such that $f(x_i) = v_i$.

Let $f$ be the map from $M$ to $M''$ in the exact sequence. For each $m \in M$, either $m$ goes to 0 or $f(m)$ is a finite sum $s_1 v_1 + s_2 v_2 + \cdots + s_n v_n$. In the first case, it is in the submodule generated by $w_1, w_2, \cdots, w_m$. In the second case, it is in the submodule generated by $x_1, x_2, \cdots, x_n$. In all cases, $m$ is in the submodule generated by $w_1, w_2, \cdots, w_m, x_1, x_2, \cdots, x_n$. Thus $M$ is finitely generated as desired.

\section{}
\subsection{Problem}
Let $A$ be a ring, $\mathfrak{a}$ an ideal contained in the Jacobson radical of $A$; let $M$ be an $A$-module and $N$ a finitely generated $A$-module, and let $u: M \rightarrow N$ be a homomorphism. If the induced homomorphism $M/\mathfrak{a}M \rightarrow N/\mathfrak{a}N$ is surjective, then $u$ is surjective.

\subsection{Solution}

\section{}
\subsection{Problem}
Let $A$ be a ring $\neq 0$. Show that $A^m \cong A^n \Rightarrow m = n$.

\subsection{Solution}
Let $\mathfrak{m}$ be a maximal ideal of $A$, let $k = A/\mathfrak{m}$, and let $\phi: A^m \rightarrow A^n$ be an isomorphism. Then $1 \otimes_k \phi: k \otimes_k A^m \rightarrow k \otimes_k A^n$ is an isomorphism of vector spaces of dimensions $m$ and $n$. Thus, $m = n$.

\section{}
\subsection{Problem}
Let $M$ be a finitely generated $A$-module and $\phi: M \rightarrow A^n$ a surjective homomorphism. Show that Ker $(\phi)$ is finitely generated.

\subsection{Solution}
Ker $(\phi) \subset M$ and $M$ is finitely generated, so Ker $(\phi)$ must be finitely generated.

\section{}
\subsection{Problem}
Let $f: A \rightarrow B$ be a ring homomorphism, and let $N$ be a $B$-module. Regarding $N$ as an $A$-module by restriction of scalars, form the $B$-module $N_B = B \otimes_A N$. Show that the homomorphism $g: N \rightarrow N_B$ which maps $y$ to $1 \otimes y$ is injective and that $g(N)$ is a direct summand of $N_B$.

\subsection{Solution}


\chapter{}

\section{}
\subsection{Problem}
Let $S$ be a multiplicatively closed subset of a ring $A$, and let $M$ be a finitely generated $A$-module. Prove that $S^{-1}M = 0$ if and only if there exists $s \in S$ such that $sM = 0$.

\subsection{Solution}
$\Leftarrow$: Let $m/t \in S^{-1}M$. Now note that $m/t  = 0/1 \Leftrightarrow mu = 0$ for some $u \in S$. But since $sM = 0$, $ms = 0$ for all $m \in M$, so setting $u = s$ establishes that $m/t = 0$, so $S^{-1}M = 0$.

$\Rightarrow$: Suppose that $S^{-1}M = 0$. Then for all $m/1 \in S^{-1}M$, we have $m/1 = 0/1$, so there exists a $u \in S$ such that $mu = 0$. Let $e_1, e_2, \cdots, e_n$ be the generators of $M$. Let their corresponding annihilators in $S$ be $u_1, u_2, \cdots, u_n$. Then $u_1 u_2 \cdots u_n$ annihilates every element of $M$. Thus, we are done.

\section{}
\subsection{Problem}
Let $\mathfrak{a}$ be an ideal of a ring $A$, and let $S = 1 + \mathfrak{a}$. Show that $S^{-1}\mathfrak{a}$ is contained in the Jacobson radical of $S^{-1}A$. 

\subsection{Solution}
I will show that $\mathfrak{a}$ is contained in every maximal ideal of $A$. Suppose not. Then there would exist $a \in \mathfrak{a}$, $m \in M$ such that $a + m = 1$. This implies that $a$ is a unit. Thus, $\mathfrak{a}$ is contained in every maximal ideal of $A$. It is also clear that $\mathfrak{a}$ and $S$ are disjoint. By the one-to-one correspondence between prime ideals of $A$ and prime ideals of $S^{-1}A$, this means that $S^{-1}\mathfrak{a}$ is contained in all the maximal ideals of $S^{-1}A$. The one-to-one correspondence applies to maximal ideals because they are disjoint from $S = 1 + \mathfrak{a}$. If $m = 1 + a$, this implies that $a$ is a unit. Therefore, $S^{-1}\mathfrak{a}$ is contained in the Jacobson radical of $S^{-1}A$, as desired.


\section{}
\subsection{Problem}
Let $A$ be a ring, let $S$ and $T$ be two multiplicatively closed subsets of $A$, and let $U$ be the image of $T$ in $S^{-1}A$. Show that the rings $(ST)^{-1}A$ and $U^{-1}(S^{-1}A)$ are isomorphic.

\subsection{Solution}
Let $f: (ST)^{-1}A \rightarrow U^{-1}(S^{-1}A)$ be defined by $f(a/st) = (a/s)/(t/1)$. This map is clearly surjective. Now suppose that $f(a_1/s_1t_1) = f(a_2/s_2t_2)$. Then $(a_1/s_1)/(t_1/1) = (a_2/s_2)/(t_2/1)$. Then there is some $a_3/s_3 \in S^{-1}A$ such that $((a_1/s_1)(t_2/1) - (a_2/s_2)(t_1/1))a_3/s_3 = 0 \Rightarrow (a_1 t_2/s_1 - a_2 t_1/s_2)a_3/s_3 = 0 \Rightarrow (a_1t_2s_2 - a_2 t_1s_1)/s_1 s_2 \cdot (a_3/s_3) = 0 \Rightarrow a_1t_2s_2a_3 - a_2 t_1 s_1 a_3 = 0 $ This means that $a_1/(s_1t_1) = a_2/(s_2t_2)$, so $f$ is injective as well. Thus, $f$ is an isomorphism, as desired.

\section{}
\subsection{Problem}
Let $f: A \rightarrow B$ be a homomorphism of rings and let $S$ be a multiplicatively closed subset of $A$. Let $T = f(S)$. Show that $S^{-1}B$ and $T^{-1}B$ are isomorphic as $S^{-1}A$-modules.

\subsection{Solution}

\section{}
\subsection{Problem}
Let $A$ be a ring. Suppose that, for each prime ideal $\mathfrak{p}$, the local ring $A_{\mathfrak{p}}$ has no nilpotent element $\neq 0$. Show that $A$ has no nilpotent element $\neq 0$. If each $A_{\mathfrak{p}}$ is an integral domain, is $A$ necessarily an integral domain?

\subsection{Solution}











\end{document}


