\documentclass[book,12pt,oneside,openany]{memoir}
\usepackage[utf8x]{inputenc}
\usepackage[english]{babel}
\usepackage{amsmath}
\usepackage{amssymb}
\usepackage{enumerate}
\usepackage{url}

% for placeholder text
\usepackage{lipsum}

\title{Solutions to \emph{Introduction to Commutative Algebra} by Atiyah and Macdonald}
\author{Aditya Gudibanda}
\begin{document}

\maketitle

\chapter{}

\section{}
\subsection{Problem}
Let $x$ be a nilpotent element of a ring $A$. Show that $1 + x$ is a unit of  $A$. Deduce that the sum of a nilpotent element and a unit is a unit.

\subsection{Solution}
Note the identity $\left(\sum_{i = 0}^k x_i \right) (1-x) =  1- x^{k+1}$. Since $x$ is nilpotent,  $x^l = 0$ for some $l \geq 1$. Thus, if we set $k = l-1$ in the identity above, we see that $1-x$ has a multiplicative inverse, and is therefore  a unit. 

To prove the original statement, given that $x$ is a nilpotent unit, it is clear that $-x$ is nilpotent as well, and by the logic above, $1 - (-x) = 1 + x$ is a unit, as desired.

Suppose $u$ is an arbitrary unit. Since $x$ is nilpotent, $u^{-1}x$ is nilpotent as well, so by the above, $u^{-1}x + 1$ is a unit. Since the product of units is a unit, $u(u^{-1}x + 1) = x + u$ is a unit, as desired.

\section{}
\subsection{Problem}
Let $A$ be a ring and let $A[x]$ be the ring of polynomials in an indeterminate $x$, with coefficients in $A$. Let $f = a_0 + a_1x + \cdots + a_n x^n \in A[x]$. Prove that

i) $f$ is a unit in $A[x] \Leftrightarrow a_0$ is a unit in $A$ and $a_1, \ldots, a_n$ are nilpotent. 

ii) $f$ is nilpotent $\Leftrightarrow a_0, a_1, \ldots, a_n$ are nilpotent.

iii) $f$ is a zero-divisor $\Leftrightarrow$ there exists $a \neq 0$ in $A$ such that $af = 0$.

iv) $f$ is said to be primitive if $(a_0, a_1, \ldots, a_n) = (1)$. Prove that if $f,g \in A[x]$, then $fg$ is primitive $\Leftrightarrow f$ and $g$ are primitive.

\subsection{Solution}
TODO

\section{}
\subsection{Problem}
Generalize the results of Exercise 2 to a polynomial ring $A[x_1, \ldots, x_r]$ in several indeterminates.

\subsection{Solution}
TODO

\section{}
\subsection{Problem}
In the ring $A[x]$, the Jacobson radical is equal to the nilradical.

\subsection{Solution}
One inclusion holds for any ring.
Suppose $1-fg$ is always a unit for all $g$.
Letting $g=x$, all coefficients of $f$ must be nilpotent by exercise $2$.

\section{}
\subsection{Problem}
Let $A$ be a ring and let $A[[x]]$ be the ring of formal power series $f = \sum_{n=0}^{\infty} a_n x^n$ with coefficients in $A$. Show that 

i) $f$ is a unit in $A[[x]] \Leftrightarrow a_0$ is a unit in $A$.

ii) If $f$ is nilpotent, then $a_n$ is nilpotent for all $n \geq 0$. Is the converse true?

iii) $f$ belongs to the Jacobson radical of $A[[x]] \Leftrightarrow a_0$ belongs to the Jacobson radical of $A$.

iv) The contraction of a maximal ideal $\mathfrak{m}$ of $A[[x]]$ is a maximal ideal of $A$, and $\mathfrak{m}$ is generated by $\mathfrak{m}^c$ and $x$.

v) Every prime ideal of $A$ is the contraction of a prime ideal of $A[[x]]$.


\subsection{Solution}
TODO
\section{}
\subsection{Problem}
A ring $A$ is such that every ideal not contained in the nilradical contains a nonzero idempotent (that is, an element $e$ such that $e^2 = e \neq 0$). Prove that the nilradical and Jacobson radical of $A$ are equal.
\subsection{Solution}
The Jacobson radical is contained in the nilradical in any ring.

Suppose $\mathfrak R$ contains a nonzero idempotent $x$.
Then $1-xx=1-x$ is a unit.
But $x(1-x)=0$ so $1-x$ is also a zero divisor.


\section{}
\subsection{Problem}
Let $A$ be a ring in which every element $x$ satisfies $x^n = x$ for some $n > 1$ (depending on $x$). Show that every prime ideal in $A$ is maximal.
\subsection{Solution}
Let $\mathfrak{a}$ be a prime ideal. Then $A/\mathfrak{a}$ is an integral domain. Suppose that $x \in A$. Then $x^n = x$ for some $n > 1$. Let $\bar{x}$ be the image of $x$ under the projection to the quotient by $\mathfrak{a}$. Then $\bar{x}^n - \bar{x} = \bar{x} (\bar{x}^{n-1} - 1) = 0$. Since $A/\mathfrak{a}$ is an integral domain, this implies that either $\bar{x} = 0$ or $\bar{x}^{n-1} = 1$. If $\bar{x}^{n-1} = 1$, then $\bar{x} \cdot \bar{x}^{n-2} = 1$, so $\bar{x}$ has an inverse. This implies that for every $\bar{x} \in A/\mathfrak{a}$, either $\bar{x} = 0$ or $\bar{x}$ has an inverse. Thus, $A/\mathfrak{a}$ is a field, so $\mathfrak{a}$ is maximal, as desired.


\section{}
\subsection{Problem}
Let $A$ be a ring $\neq 0$. Show that the set of prime ideals of $A$ has minimal elements with respect to inclusion.
\subsection{Solution}
For any prime $\mathfrak p$, $A/\mathfrak p$ is an integral domain.
For any $x\in A$, let $\bar{x}=x+\mathfrak p$.
$\bar{x}(\bar{x}^{n-1}-1)=\bar{x}^n-\bar{x}=0$.
For $x\notin\mathfrak p$, $\bar{x}^{n-1}=1$, so $\bar{x}$ is a unit.
As $A/\mathfrak p$ is a field, $\mathfrak p$ is maximal.


\section{}
\subsection{Problem}
Let $\mathfrak{a}$ be an ideal $\neq (1)$ in a ring $A$. Show that $\mathfrak{a} = r(\mathfrak{a}) \Leftrightarrow \mathfrak{a}$ is an intersection of prime ideals.
\subsection{Solution}

$\Leftarrow$: Suppose $\mathfrak{a}$ is an intersection of prime ideals. Clearly, $\mathfrak{a} \subset r(\mathfrak{a})$. Suppose $x \notin \mathfrak{a}$. Then there is some prime ideal $\mathfrak{p}$ which $x$ is not in, but in which $\mathfrak{a}$ is contained. Thus, no power of $x$ will ever be in $\mathfrak{p}$, so no power of $x$ will ever be in $\mathfrak{a}$. Thus, $\mathfrak{a} = r(\mathfrak{a})$, as desired.

$\Rightarrow$:  Simply apply Proposition 1.14 - the radical of an ideal $\mathfrak{a}$, in this case $\mathfrak{a}$, is the intersection of the prime ideals which contain $\mathfrak{a}$.
\section{}
\subsection{Problem}
Let $A$ be a ring, $\mathcal{R}$ its nilradical. Show that the following are equivalent:

i) $A$ has exactly one prime ideal

ii) every element of $A$ is either a unit or nilpotent

iii) $A/\mathcal{R}$ is a field
\subsection{Solution}
.

i $\Rightarrow$ iii: Suppose $A$ has exactly one prime ideal. Let $\mathfrak{m}$ be the maximal ideal of $A$. Then this must be the unique prime ideal, so it is also the nilradical $\mathcal{R}$. Thus, $\mathcal{R}$ is maximal, so $A/\mathcal{R}$ is a field.

iii $\Rightarrow$ ii: Suppose $A/\mathcal{R}$ is a field. Then $R$ is maximal, so all the elements in $A - \mathcal{R}$ are units. All the elements within $\mathcal{R}$ are obviously nilpotent. Thus, every element of $A$ is either a unit or nilpotent, as desired.

ii $\Rightarrow$ i: Let $\mathfrak{p}$ be a prime ideal. Clearly it must contain all the nilpotent units and can contain none of the units. Thus, every prime ideal must contain exactly all of the nilpotents. Thus, $A$ has exactly one prime ideal.



\section{}
\subsection{Problem}
A ring $A$ is Boolean if $x^2 = x$ for all $x \in A$. In a Boolean ring $A$, show that 

i) $2x = 0$ for all $x \in A$

ii) every prime ideal $\mathfrak{p}$ is maximal, and $A/\mathfrak{p}$ is a field with two elements

iii) every finitely generated ideal in $A$ is principal.
\subsection{Solution}

.

i) $(x +1)^2 = x^2 + 2x + 1 = x + 2x + 1 = x + 1 \Rightarrow 2x = 0$, as desired.

ii) Let $\mathfrak{p}$ be a prime ideal. Then every element in $A/\mathfrak{p}$ satisfies $\bar{x}^2 = \bar{x}$. Since $A/\mathfrak{p}$ is an integral domain, this means that $\bar{x} = 0$ or $\bar{x} = 1$. Thus, $A/\mathfrak{p}$ has only two elements and is hence a field, so $\mathfrak{p}$ is maximal, as desired.

iii) It suffices to be able to reduce the ideal $(a, b)$ and show that it is equivalent to another ideal $(c)$. To this end, let $c = ab + a + b$. It is clear that $c \in (a,b)$. Next, $ca = a^2b + a^2 + ab = ab + a + ab = a$. Next, $cb = ab^2 + ab + b^2 = ab + ab + b = b$. Thus, $(a,b) = (c)$. So given any finite set of generators, we can reduce these to one generator, by induction, by reducing the number of generators one by one.


\section{}
\subsection{Problem}
A local ring contains no idempotent $\neq 0,1$.
\subsection{Solution}
Let $x \in A$, the local ring. Let $\mathfrak{m}$ be the maximal ideal of $A$. Suppose $x$ is idempotent. Then in $A/\mathfrak{m}$, $\bar{x}^2 = \bar{x} \Rightarrow \bar{x} ( \bar{x} - 1) = 0$. Since $A/\mathfrak{m}$ is a field, this implies that either $\bar{x} = 0$ or $\bar{x} = 1$.

If $\bar{x} = 1$, then $x$ is a unit, so if $x^2 = x$, then $x(x - 1) = 0$, so $x$ is either $0$ or $1$. $0$ is not a unit, so $x = 1$.

If $\bar{x} = 0$, then $x \in \mathfrak{m}$, so $x - 1$ is a unit. Since $x(x - 1) = 0$, this means that $x = 0$.

Thus, $x$ is either $0$ or $1$, as desired.


\section{}
\subsection{Problem}
Let $K$ be a field and let $\Sigma$ be the set of all irreducible monic polynomials $f$ in one indeterminate with coefficients in $K$. Let $A$ be the polynomial ring over $K$ generated by indeterminates $x_f$, one for each $f \in \Sigma$. Let $\mathfrak{a}$ be the ideal of $A$ generated by the polynomials $f(x_f)$ for all $f \in \Sigma$. Show that $\mathfrak{a} \neq (1)$.

Let $\mathfrak{m}$ be the maximal ideal of $A$ containing $\mathfrak{a}$, and let $K_1 = A/\mathfrak{m}$. Then $K_1$ is an extension field of $K$ in which each $f \in \Sigma$ has a root. Repeat the construction with $K_1$ in place of K, obtaining a field $K_2$, and so on. Let $L = \cup_{n=1}^{\infty} K_n$. Then $L$ is a field in which each $f \in \Sigma$ splits completely into linear factors. Let $\bar{K}$ be the set of all elements of $L$ which are algebraic over $K$. Then $\bar{K}$ is an algebraic closure of $K$.
\subsection{Solution}
TODO


\section{}
\subsection{Problem}
In a ring $A$, let $\Sigma$ be the set of all ideals in which every element is a zero-divisor. Show that the set $\Sigma$ has maximal elements and that every maximal element of $\Sigma$ is a prime ideal. Hence the set of zero-divisors of $A$ is a union of prime ideals.
\subsection{Solution}

If $(x)+\mathfrak m$ contains a non-zero-divisor $z_1$ and
$(y)+\mathfrak m$ contains a non-zer-divisor $z_2$ then
$z_1 z_2$ is not a zero divisor and is in $(xy)+\mathfrak m$.

\section{}
\subsection{Problem}
Let $A$ be a ring and let $X$ be the set of all prime ideals of $A$. For each subset $E$ of $A$, let $V(E)$ denote the set of all prime ideals of $A$ which contain $E$. Prove that

i) if $\mathfrak{a}$ is the ideal generated by $E$, then $V(E) = V(\mathfrak{a}) = V(r(\mathfrak{a}))$.

ii) $V(0) = X, V(1) = \emptyset$.

iii) If $(E_i)_{i \in I}$ is any family of subsets of $A$, then \[V (\cup_{i \in I} E_i ) = \cap_{i \in I} V(E_i).\]

iv) $V(\mathfrak{a} \cap \mathfrak{b}) = V(\mathfrak{a}\mathfrak{b}) = V(\mathfrak{a}) \cup V(\mathfrak{b})$ for any ideals $\mathfrak{a}, \mathfrak{b}$ of $A$.

\subsection{Solution}
.

i) If $\mathfrak{p} \in V(\mathfrak{a})$, then $\mathfrak{a} \subset \mathfrak{p}$, so $E \subset \mathfrak{p}$. Thus, $\mathfrak{p} \in V(E)$. So $V(\mathfrak{a}) \subset V(E)$. For the other direction, if $\mathfrak{p} \in V(E)$, then $E \subset \mathfrak{p}$. So $\mathfrak{p}$ must contain all the elements in the ideal generated by $E$, which is $\mathfrak{a}$. Thus, $\mathfrak{a} \subset \mathfrak{p}$, so $\mathfrak{p} \in V(\mathfrak{a})$. Thus, $V(E) = V(\mathfrak{a})$. Next, let $\mathfrak{p} \in V(\mathfrak{a})$. Let $x \in r(\mathfrak{a}) - \mathfrak{a}$, with $n \in \mathbb{N}$ being the minimal exponent such that $x^n \in \mathfrak{a}$. Clearly, $n > 1$. Then consider the fact that $x \cdot x^{n-1} \in \mathfrak{a}$. By the definition of $n$, both $x$ and $x^{n-1}$ are not in $\mathfrak{a}$. But $x^{n} \in \mathfrak{a} \Rightarrow x^n \in \mathfrak{p}$, so either $x$ or $x^{n-1}$ must be in $\mathfrak{p}$. If it is $x^{n-1}$, then we perform the same argument with $x$ and $x^{n-2}$, etc. until we reach the conclusion that $x \in \mathfrak{p}$. Thus, $r(\mathfrak{a}) \subset \mathfrak{p}$, so $\mathfrak{p} \in V(r(\mathfrak{a}))$. For the other direction, $V(r(\mathfrak{a})) \subset V(\mathfrak{a})$ is obvious because $\mathfrak{a} \subset r(\mathfrak{a})$. Thus, $V(E) = V(\mathfrak{a}) = V(r(\mathfrak{a}))$, as desired.

ii) $V(0) = X$, because every prime ideal contains $0$. $V(1) = \emptyset$, because if an ideal contains $1$, then it contains all of $A$ and it cannot be prime.

iii) If $\mathfrak{p} \in V(\cup_{i \in I} E_i)$, then $\mathfrak{p} \in V(E_i)$, $i \in I$, so $p \in \cap_{i \in I} V(E_i)$. If $\mathfrak{p} \in \cap_{i \in I} V(E_i)$, then $\cap_{i \in I} E_i \in \mathfrak{p}$, so $E_i \in \mathfrak{p}$, $i \in I$, so $\cup_{i \in I} E_i \subset \mathfrak{p}$, so $\mathfrak{p} \in V(\cup_{i \in I} E_i)$. Thus, $V (\cup_{i \in I} E_i ) = \cap_{i \in I} V(E_i)$, as desired.

iv) Since $\mathfrak{ab} \subset \mathfrak{a} \cap \mathfrak{b}$, it is clear that $V(\mathfrak{a} \cap \mathfrak{b}) \subset V(\mathfrak{ab})$. Now suppose $\mathfrak{p} \in V(\mathfrak{ab})$. Now suppose $x \in \mathfrak{a} \cap \mathfrak{b}$. Then clearly $x^2 \in \mathfrak{a}\mathfrak{b}$. So since $\mathfrak{p}$ is prime, $x \in \mathfrak{p}$. Thus, $V(\mathfrak{ab}) \in V(\mathfrak{a} \cap \mathfrak{b})$. Thus, $V(\mathfrak{ab}) = V(\mathfrak{a} \cap \mathfrak{b})$. This could also have been proven by noting that $r(\mathfrak{ab})  = r(\mathfrak{a} \cap \mathfrak{b})$ and using part i. Next, let $\mathfrak{p} \in V(\mathfrak{a} \cap \mathfrak{b})$. Suppose $\mathfrak{p}$ was a prime ideal such that $\mathfrak{a} - \mathfrak{p} \neq \emptyset, \mathfrak{b} - \mathfrak{p} \neq \emptyset$. Let $x \in \mathfrak{a} - \mathfrak{p}$ and $y \in \mathfrak{b} - \mathfrak{p}$. Then $xy \in \mathfrak{ab} \subset \mathfrak{p}$, but $x,y \notin \mathfrak{p}$, a contradiction. Thus, $\mathfrak{p}$ must contain either $\mathfrak{a}$ or $\mathfrak{b}$, so $\mathfrak{p} \in V(\mathfrak{a}) \cup V(\mathfrak{b})$. Next, suppose $\mathfrak{p} \in V(\mathfrak{a}) \cup V(\mathfrak{b})$. WLOG, suppose $\mathfrak{p} \in V(\mathfrak{a})$. Then clearly $\mathfrak{a} \cap \mathfrak{b} \subset \mathfrak{p}$, so $\mathfrak{p} \in V(\mathfrak{a} \cap \mathfrak{b})$. Thus, $V(\mathfrak{a} \cap \mathfrak{b}) = V(\mathfrak{a}\mathfrak{b}) = V(\mathfrak{a}) \cup V(\mathfrak{b})$, as desired.


\section{}
\subsection{Problem}
Draw pictures of Spec $\mathbb{Z}$, Spec $\mathbb{R}$, Spec $\mathbb{C}[x]$, Spec $\mathbb{R}[x]$, Spec $\mathbb{Z}[x]$.
\subsection{Solution}
TODO


\section{}
\subsection{Problem}
For each $f\in A$, let $X_f$ denote the complement of $V(f)$ in
$X=\text{Spec}(A)$.
The sets $X_f$ are open.
Show that they form a basis of open sets for the Zariski topology,
and that
\begin{enumerate}[i)]
\item $X_f \cap X_g = X_{fg}$
\item $X_f=\varnothing \iff f$ is nilpotent
\item $X_f=X \iff f$ is a unit
\item $X_f=X_g\iff r((f))=r((g))$
\item $X$ is quasi-compact.
\item each $X_f$ is quasi-compact.
\item An open subset of $X$ is quasi-compact if and only if it is the
finite union of sets $X_f$.
\end{enumerate}
\subsection{Solution}
The complement of $V(E)$ is $\bigcup_{f\in E} X_f$ by De Morgan's laws.
$X_f \cap X_g = V((f)) \cup V((g)) = V((fg))$.

The nilpotent elements are precisely those contained in every prime ideal.

The units are precisely those that generate $(1)$ and $V(1)=\varnothing$.

If $X_f=X_g$ then $V(r((f)))=V(r((g)))$.
Any prime ideal containing $r((f))$ contains $r((g))$ and vice versa
so the intersection of these is $r((f))=r((g))$.

Without loss of generality, $X$ is covered by $X_{f}$ for $f\in S$.
$V(S)=\varnothing$ so $S$ is not contained in any maximal ideal and
must generate $(1)$.
So $1=\sum_{i\in J} g_i f_i$ for some finite subset $J$. and so
$(f_i)_{i\in J}=(1)$ and $X_{f_i}$ cover $X$.

The intersection of prime ideals containing $S$ is the radical $r(S)$.
$V(S)=V((f))$ so $(f)\subseteq \mathfrak p$ for all $S\subseteq \mathfrak p$
which implies $(f)\subseteq r(S)$.
So $f^n=\sum_{i\in J} g_i f_i$ for a finite subset $J$ and some $n>0$.
So $\{X_{f_i}:i\in J\}$ covers $X_f$.

If an open subset of $X$ is not a finite union of $X_f$ then any
cover using $X_f$ must be the infinite union of $X_f$.
If it is a finite union of $X_f$ then since each $X_f$ is individually
compact we must have that every open cover has a finite subcover
for each $X_f$ in the union, which is a finite subcover of the open
subset.

\section{}
\subsection{Problem}
Show that
\begin{enumerate}[i)]
\item The set $\{x\}$ is closed.
\item $\overline{\{x\}}=V(\mathfrak p_x)$
\item $y\in \overline{\{x\}}\iff \mathfrak p_x \subseteq \mathfrak p_y$
\item $X$ is a $T_0$-space.
\end{enumerate}
\subsection{Solution}
If $\mathfrak p_x$ is maximal then there is an open set containing just
$\{x\}$.
If there is an open set containing just $\{x\}$ then $x$ is not a subset
of any other prime ideal and is therefore maximal.

The smallest closed set containing $\{x\}$ consists of the prime ideals
containing $x$.

$y$ contains $x$ if and only if $y$ is in the set of all prime ideals
satisfying this property.

Every distinct pair of points satisfy $x \not \subseteq y$ without loss of
generality.
In this case $y \not\in \overline{\{x\}}$.


\section{}
\subsection{Problem}
A topological space $X$ is said to be irreducible if $X \neq \emptyset$ and if every pair of non-empty open sets in $X$ intersect, or equivalently if every non-empty open set is dense in $X$. Show that Spec $A$ is irreducible if and only if the nilradical of $A$ is a prime ideal.
\subsection{Solution}
If the nilradical is prime then every open set contains the nilradical,
as it is a point in our space.

For $fg\in \mathfrak N$ with $f,g\notin \mathfrak N$, $X_{fg}=\varnothing$
but $X_f$ and $X_g$ are nonempty.


\section{}
\subsection{Problem}
Let $X$ be a topological space.
\begin{enumerate}[i)]
\item If $Y$ is an irreducible subspace of $X$ then the closure
$\overline{Y}$ in $X$ is irreducible.
\item Every irreducible subspace of $X$ is contained in a maximal
irreducible subspace.
\item The maximal irreducible subspaces are closed and cover $X$.
What are the irreducible components of a Hausdorff space?
\item If $A$ is a ring and $X=\text{Spec}(A)$, then the
irreducible components of $X$ are the closed sets $V(\mathfrak p)$,
where $\mathfrak p$ is a minimal prime ideal of $A$.
\end{enumerate}
\subsection{Solution}
For $A,B$ in the closure of $Y$, $A\cap Y$ and $B \cap Y$ are open.
The nonempty case would have nonempty intersection, so without
loss of generality $A\cap Y$ is empty.
Then $\overline{Y}\setminus Y$ has nonempty interior, which is impossible.

The union of a chain of irreducible subspaces is irreducible.
To see this, every point in the union of a chain is contained in some
element of that chain.
Let $X_i$ be such that $A$ is nonempty and open and
$X_j$ be such that $B$ is nonempty and open.
Without loss of generality, $X_i \subseteq X_j$, so $A$ and $B$ are nonempty
and open in $X_j$ and have nonempty intersection there, and therefore
nonempty intersection in the union.
By Zorn's lemma, there is a maximal irreducible subspace.

Every singleton is an irreducible subspace.
Every point's singleton is contained in some maximal irreducible subspace,
which proves covering.
Moreover, the maximal irreducible subspaces must be equal to their closure
and therefore are closed.

The irreducible components of a Hausdorff space are singletons, as every
two distinct points have disjoint open sets containing them.
This irreducibility in fact is equivalent to the Hausdorff condition.

Every two open sets in $V(\mathfrak p)$ must both contain the point
$\mathfrak p$.
Every nonzero ideal contains a minimal prime ideal, so these closed
irreducible subspaces are covering.
If $V(\mathfrak p) \cup V(\mathfrak p')$ is irreducible then
$V(\mathfrak{pp'})$ is irreducible, but for distinct minimal prime
ideals, this product is $0$.

\section{}
\subsection{Problem}
Let $\phi:A\to B$ be a ring homomorphism.
Let $X=\text{Spec}(A)$ and $Y=\text{Spec}(B)$.
If $\mathfrak q\in Y$ then $\phi^{-1}(\mathfrak q)$ is a prime ideal of $A$.
Consider the induced mapping $\phi^* : Y \to X$.
\begin{enumerate}[i)]
\item If $f\in A$ then $\phi^{*-1}(X_f)=Y_{\phi(f)}$.
\item If $\mathfrak a$ is an ideal of $A$ then
$\phi^{*-1}(V(\mathfrak a))=V(\mathfrak a^e)$.
\item If $\mathfrak b$ is an ideal of $B$ then
$\overline{\phi^*(V_b)}=V(b^c)$.
\item If $\phi$ is surjective, then $\phi^*$ is a homeomorphism of $Y$
onto the closed subset $V(\ker(\phi))$ of $X$.
\item $\phi^*(Y)$ is dense in $X \iff \ker(\phi)\subseteq \mathfrak N$.
\item Let $\psi:B\to C$ be another ring homomorphism.
$(\psi\circ\phi)^*=\phi^*\circ\psi^*$.
\item Let $A$ be an integral domain with just one non-zero prime ideal
$\mathfrak p$, and let $K$ be the field of fractions in $A$.
Let $B=(A/\mathfrak p)\times K$.
Define $\phi : A \to B$ by $\phi(x)=(\bar{x},x)$,
where $\bar{x}$ is the image of $x$ in $A/\mathfrak p$.
Show that $\phi^*$ is bijective but not a homeomorphism.
\end{enumerate}
\subsection{Solution}
$f\in \phi^{-1}(\mathfrak q) \iff \phi(f)\in \mathfrak q$.
So $\phi^{-1}(\mathfrak q)\in V(f)\iff \mathfrak q\in V(\phi(f))$.

$\phi^{*-1}(V(\mathfrak a)) = \bigcap_{f\in \mathfrak a} \phi^{*-1}(V(f))
= \bigcap_{f\in \mathfrak a} V(\phi(f))$
So this maps to $V(\mathfrak a^e)$, since it is the smallest prime set
containing $\phi(f)$ for $f\in\mathfrak a$.

A prime ideal contains $r(\mathfrak b)^c$ iff it contains
the contraction of every prime ideal containing $\mathfrak b$,
which is the image of $V(\mathfrak b)$ under $\phi^*$.
The result follows by $V(\bigcap A)=\bar{A}$.

In the map $\mathfrak b\mapsto b^c$ and
inverse map $\mathfrak a \mapsto \mathfrak a^e$,
Continuity of both maps follows by the second and third parts.
The set of contracted ideals is the set of prime ideals
containing $\ker(\phi)$.
So the composition of the maps is the identity.

$\ker(\phi)\subseteq\mathfrak N\iff V(\ker(\phi))=X$ which implies
$\phi^*$ is a homeomorphism.
Density implies that no prime ideal is contained in
$X\setminus V(\ker(\phi))$.

\begin{align*}
  (\psi\circ\phi)^*(\mathfrak q)&=(\psi\circ\phi)^{-1}(\mathfrak q)\\
  &=\phi^{-1}(\psi^{-1}(\mathfrak q))\\
  &=(\phi^*\circ\psi^*)(\mathfrak q)
\end{align*}

There are two prime ideals in $B$ which are $(1)\times(0)$ and
$(0)\times(1)$ since $(0)\times(0)$ is the product of these two prime
ideals and $(1)\times(1)$ is the entire space.
$\phi^*((0)\times(1))$ is the prime ideal $\mathfrak p$.
$\phi^*((1)\times(0))$ is the prime ideal $0$.
Since there are only two elements each, $\phi^*$ is a bijection.
$\phi^*(V((1)\times(0)))$ contains just $0$ and not $\mathfrak p$, which
is not closed.
So $\phi^*$ is not an open mapping.
\section{}
\subsection{Problem}
Let $A$ be the direct product of rings $A_i$.
Show that Spec$(A)$ is the disjoint union of open subspaces $X_i$
where $X_i$ is canonically homeomorphic with Spec$(A_i)$.

Conversely, let $A$ be any ring.
Show that the following are equivalent:
\begin{enumerate}[i)]
\item $X=\text{Spec}(A)$ is disconnected.
\item $A\simeq A_1\times A_2$ where neither of the rings $A_1,A_2$
is the zero ring.
\item $A$ contains an idempotent $\ne 0,1$.
\end{enumerate}
\subsection{Solution}
Let $V(\mathfrak a)$ and $V(\mathfrak b)$ be closed and open.
Then $\mathfrak a+\mathfrak b=1$.
So $a+b=1$ with $a\in\mathfrak a$ and $b\in\mathfrak b$.
$V(\mathfrak a\mathfrak b)$ is empty so is contained in the nilradical.
So $(ab)^n=0$ and $(a^n)+(b^n)=1$ so we can take $e\in(a^n)$ and
$1-e\in(b^n)$.
$e-e^2=0$ so $e$ is idempotent.

$(1,0)^2=(1,0)$ is a nonzero idempotent.

If $x$ is idempotent, then $(1-x)^2=1-x$ is also idempotent.
$(x)\cap(1-x)=(0)$ because the product is $(0)$.
Nonzero $\mathfrak a$ has nonzero intersection with exactly one of
$(x)$ or $(1-x)$.
$\phi:A\to A/(e)\times A/(1-e)$ is a bijective homomorphism.

$e^2=e$ and $(1-e)^2=1-e$.
$(e)$ and $(1-e)$ are coprime with $(0)$ intersection.
$V((e))\cup V((1-e))=V((0))=X$.
$V((e))\cap V((1-e))=V((1))=\varnothing$.
\section{}
\subsection{Problem}
Let $A$ be a Boolean ring and let $X=\text{Spec}(A)$.
\begin{enumerate}[i)]
\item For $f\in A$, the lset $X_f$ is both open and closed in $X$.
\item Let $f_1,\dots,f_n\in A$.  $X_{f_1}\cup\dots\cup X_{f_n}=X_f$
for some $f$.
\item The sets $X_f$ are the only subsets of $X$ which are both
open and closed.
\item $X$ is a compact Hausdorff space.
\end{enumerate}
\subsection{Solution}
$X_f$ and $X_{1-f}$ are disjoint since every maximal ideal contains
one or the other.

Every finitely generated ideal is principal.
$(f_1, \dots, f_n)=(f)$ for some $f$ so
$\bigcup_i X_{f_i}=X_f$.

$X$ is quasi-compact, so $Y \subset X$ is quasi-compact.
$Y$ is open, so it can be written as the union of finitely
many $X_f$.
So $Y$ is equal to $X_g$ for some $g$.

If for every $f$, $(f) \subset g$ iff $(f) \subset h$
then $g$ and $h$ are equal.
Otherwise, $g\in X_f$ and $h\in X_{1-f}$ for some $f$, disjoint
open sets.
\section{}
\subsection{Problem}
Let $L$ be a Boolean lattice.
Define $a+b=(a\land b')\lor(a'\land b)$ and $ab=a\land b$.
$L$ is a Boolean ring.
Conversely, show every Boolean ring induces a Boolean lattice.
\subsection{Solution}
$\land$ is associative, idempotent, commutative, and unital.

Verification of commutativity and associativity is a matter
of computation.

$a+0=(a\land 1) \lor 0=a$.

Distributivity follows from distributivity in the lattice
and de Morgan's laws.

(the converse) $1$ is the unit, $0$ is the additive identity.
Reflexivity of $\le$ is idempotency of $\cdot$.
Transitivity of $\le$ is simply multiplication.
$a=ab$ and $b=ab$ implies $a=b$.
The complement of $a$ is $(1-a)$.
$\land$ distributes over $\lor$ and vice versa.
\section{}
\subsection{Problem}
From the last two exercises deduce Stone's theorem, that
every Boolean lattice is isomorphic to the lattice of
open-and-closed subsets of some compact Hausdorff
topological space.
\subsection{Solution}
The open-and-closed subsets of the spectrum of the corresponding
Boolean ring form a Hausdorff space such that the open and closed
subsets are precisely the sets $X_f$ for $f\in A$.

$X_f \le X_g$ if and only if $f=fg$ which induces the Boolean lattice.
\section{}
\subsection{Problem}
Let $f\in C(X)$.
Let $U_f=\{x\in X:f(x)\ne 0\}$.
Let $\tilde{X}$ be the set of all maximal ideals.
Let $\tilde{U}_f=\{\mathfrak m\in\tilde{X}:f\notin\mathfrak m\}$.
Let $\mathfrak m_x$ be the set of all functions $f$ such that $f(x)=0$.
The map $\mu:X\to\tilde{X}$ sends $x\mapsto\mathfrak m_x$.
Show that $\mu(U_f)=\tilde{U}_f$.
\subsection{Solution}
$x\in U_f\iff f(x)\ne0\iff f\notin m_x\iff \mu(x)\in\tilde{U}_f$.
\section{}
\subsection{Problem}
Regular mappings acting on $k$-algebra homomorphisms by right composition
put regular mappings in one-to-one correspondence with $k$-algebra
homomorphisms.
\subsection{Solution}
\section{}
\subsection{Problem}
\subsection{Solution}


\chapter{}

\section{}
\subsection{Problem}
Show that $\left( \mathbb{Z}/m\mathbb{Z} \right) \otimes_{\mathbb{Z}}  \left( \mathbb{Z}/n\mathbb{Z} \right) = 0$ if $m,n$ are coprime.
\subsection{Solution}
The tensor product is generated by all possible tensor products of pairs of generators from the two rings we are tensoring. Both rings are cyclic and are generated by 1. So the tensor product is generated by $1 \otimes 1$.  Now, note that $m (1 \otimes 1) = m \otimes 1 = 0 \otimes 1 = 0$, and $n (1 \otimes 1) = 1 \otimes n = 1 \otimes 0 = 0$. Thus, the order of $1 \otimes 1$ must divide both $m$ and $n$. However, since $m$ and $n$ are coprime, the only such positive number is 1. Thus, $1 \otimes 1$ has order 1, so $\left( \mathbb{Z}/m\mathbb{Z} \right) \otimes_{\mathbb{Z}}  \left( \mathbb{Z}/n\mathbb{Z} \right)$ is a ring that contains only one element, which is therefore 0, as desired.

\section{}
\subsection{Problem}
Let $A$ be a ring, $\mathfrak{a}$ an ideal, $M$ an $A$-module. Show that $(A/\mathfrak{a}) \otimes_{A} M$ is isomorphic to $M/\mathfrak{a}M$.

\subsection{Solution}
Consider the exact sequence $0 \rightarrow \mathfrak{a} \rightarrow A \rightarrow A/\mathfrak{a} \rightarrow 0$. We now tensor this with $M$ to get $0 \rightarrow \mathfrak{a} \otimes_{A} M \rightarrow A \otimes_{A} M \rightarrow A/\mathfrak{a} \otimes_{A} M \rightarrow 0$. By the first isomorphism theorem, we therefore have that $A/\mathfrak{a} \otimes_{A} M \cong A \otimes_{A} M / \mathfrak{a} \otimes_A M$. Clearly, $A \otimes_A M \cong M$ and $\mathfrak{a} \otimes_{A} M \cong \mathfrak{a} M$, so we have $A/\mathfrak{a} \otimes_{A} M \cong M/\mathfrak{a}M$, as desired.

\section{}

\subsection{Problem}
Let $A$ be a local ring, $M$ and $N$ finitely generated $A$-modules. Prove that if $M \otimes N = 0$, then $M = 0$ or $N = 0$.

\subsection{Solution}
Let $\mathfrak{m}$ be the maximal ideal of $A$ and let $k = A/\mathfrak{m}$ be the residue field. Then $M_k = k \otimes_A M \cong M/\mathfrak{m}M$ by 3.2. Since $A$ is a local ring, $\mathfrak{m}$ is the only maximal ideal of $A$, so by Nakayama's lemma, if $M_k = 0$, then $M = 0$. 

Since $M \otimes_{A} N = 0$, $(M \otimes_{A} N)_k = M_k \otimes_{A} N_k = 0$. Since $M_k$ and $N_k$ are both vector spaces over a field, this implies that either $M_k = 0$ or $N_k = 0$, which implies that either $M = 0$ or $N= 0$, as desired.

\section{}
\subsection{Problem}
Let $M_i (i \in I)$ be any family of $A$-modules, and let $M$ be their direct sum. Prove that $M$ is flat $\Leftrightarrow$ each $M_i$ is flat.

\subsection{Solution}
$\Rightarrow:$ Suppose that $M$ is flat. The maximal ideals of $M$ are $\{\times_{i \neq j} M_i  \times \mathfrak{m}_j | j \in I, \mathfrak{m_j} \in M_j\}$, where $\mathfrak{m}_j$ are the maximal ideals of $M_j$. The quotients of all these maximal ideals must also be flat. These are all $M_j/\mathfrak{m}_j$, over all $j \in I$ and over all $\mathfrak{m}_j$ which are maximal in $M_j$. For a specific $j$, this implies that all the maximal ideals are flat. This implies that $M_j$ is flat. So this direction is done.

$\Leftarrow:$ Just go the other direction.


\section{}
\subsection{Problem}
Let $A[x]$ be the ring of polynomials in one indeterminate over a ring $A$. Prove that $A[x]$ is a flat $A$-algebra.
\subsection{Solution}
$A[x]$ is just the infinite direct sum of $A$, considered as an $A$-module, so $A[x]$ is flat if $A$ is flat as an $A$-algebra, by problem 2.4. But for any $A$-module $M$, $M \otimes_A A \cong M$, so clearly $A$ is flat. Thus, $A[x]$ is a flat $A$-algebra.


\section{}
\subsection{Problem}
For any $A$-module, let $M[x]$ denote the set of all polynomials in $x$ with coefficients in $M$, that is to say expressions of the form \[m_0 + m_1x + \cdots + m_rx^r (m_t \in M)\]

Defining the product of an element of $A[x]$ and an element of $M[x]$ in the obvious way, show that $M[x] \cong A[x] \otimes_{A} M$.

\subsection{Solution}
From Proposition 2.14, we know that $(M + N) \otimes P \cong (M \otimes P) + (N \otimes P)$. We can consider $M[x]$ to be an infinite direct sum of $M$, and $A[x]$ to be an infinite direct sum of $A$. From Proposition 2.14, we know that when $M$ is a module of $A$, $A \otimes M \cong M$. These two facts immediately imply the desired conclusion, $M[x] \cong A[x] \otimes_A M$.


\section{}
\subsection{Problem}
Let $\mathfrak{p}$ be a prime ideal of $A$. Show that $\mathfrak{p}[x]$ is a prime ideal in $A[x]$.

\subsection{Solution}
Let $f,g \in A[x] - \mathfrak{p}[x]$. Let \[f = a_0 + a_1x + a_2x^2 + \cdots a_n x^n\] and let \[g = b_0 + b_1 x + b_2 x^2 + \cdots + b_m x^m\]

Suppose $a_i$ is the coefficient with smallest index that is not in $\mathfrak{p}$, and let $b_j$ be defined likewise. These must both exist because $f,g \notin \mathfrak{p}[x]$. Then consider the coefficient of $x^{i+j}$ in $fg$. WLOG, let $i < j$. Then the coefficient of $x^{i+j}$ in $fg$ is \[\sum_{r = 0}^{i-1} a_r b_{i+j-r} + a_i b_j + \sum_{s = 0}^{j-1} a_{i+j-s}b_s\]

Both the sums are in $\mathfrak{p}$ because $a_r \in \mathfrak{p}$ for $r < i$ and $b_s \in \mathfrak{p}$ for $s < j$, by the definitions of $i$ and $j$. But $a_i$ and $b_j$ are not in $\mathfrak{p}$, and $\mathfrak{p}$ is prime, so $a_i b_j$ is not in $\mathfrak{p}$. Thus, $fg$ is not in $\mathfrak{p}[x]$. Thus, $\mathfrak{p}[x]$ is prime in $A[x]$, as desired.

\section{}
\subsection{Problem}

i) If $M$ and $N$ are flat $A$-modules, then so is $M \otimes_{A} N $.

ii) If $B$ is a flat $A$-algebra and $N$ is a flat $B$-module, then $N$ is flat as an $A$-module.

\subsection{Solution}

i) Let $0 \rightarrow B \rightarrow C \rightarrow D \rightarrow 0$ be an exact sequence of $A$-modules. Since $M$ is flat, \[0 \rightarrow B \otimes_A M \rightarrow C \otimes_A M \rightarrow D \otimes_A M \rightarrow 0\] is exact. Since $N$ is flat, 
\[0 \rightarrow (B \otimes_A M) \otimes_A N \rightarrow (C \otimes_A M) \otimes_A N \rightarrow (D \otimes_A M) \otimes_A N \rightarrow 0\] is also exact. By Proposition 2.14, $(M \otimes N) \otimes P \cong M \otimes (N \otimes P)$, so therefore we have 
\[0 \rightarrow B \otimes_A (M \otimes_A N) \rightarrow C \otimes_A (M \otimes_A N) \rightarrow D \otimes_A (M \otimes_A N) \rightarrow 0\] is exact. Thus, $M \otimes_A N$ is flat, as desired.

ii) Let $0 \rightarrow C \rightarrow D \rightarrow E \rightarrow 0$ be an exact sequence of $A$-modules. Since $B$ is a flat $A$-algebra, \[0 \rightarrow C \otimes_A B \rightarrow D \otimes_A B \rightarrow E \otimes_A B \rightarrow 0\] is exact. Since $N$ is a flat $B$-module, \[0 \rightarrow C \otimes_A B \otimes_B N \rightarrow D \otimes_A B \otimes_B N \rightarrow E \otimes_A B \otimes_B N \rightarrow 0\] 

But $B \otimes_B N \cong N$, so we get 
 \[0 \rightarrow C \otimes_A  N \rightarrow D \otimes_A N \rightarrow E \otimes_A  N \rightarrow 0\] 

which implies that $N$ is flat as an $A$-module, as desired.
\section{}
\subsection{Problem}
Let $0 \rightarrow M' \rightarrow M \rightarrow M'' \rightarrow 0$ be an exact sequence of $A$-modules. If $M'$ and $M''$ are finitely generated, then so is $M$.

\subsection{Solution}
Let $u_1, u_2, \cdots, u_m$ be the generators of $M'$ and let $v_1, v_2, \cdots v_n$ be the generators of $M''$. Let $w_i = f(u_i), 1 \leq i \leq m$. Since $M$ surjects onto $M''$, for each of the $v_i$, there is an $x_i$ such that $f(x_i) = v_i$.

Let $f$ be the map from $M$ to $M''$ in the exact sequence. For each $m \in M$, either $m$ goes to 0 or $f(m)$ is a finite sum $s_1 v_1 + s_2 v_2 + \cdots + s_n v_n$. In the first case, it is in the submodule generated by $w_1, w_2, \cdots, w_m$. In the second case, it is in the submodule generated by $x_1, x_2, \cdots, x_n$. In all cases, $m$ is in the submodule generated by $w_1, w_2, \cdots, w_m, x_1, x_2, \cdots, x_n$. Thus $M$ is finitely generated as desired.

\section{}
\subsection{Problem}
Let $A$ be a ring, $\mathfrak{a}$ an ideal contained in the Jacobson radical of $A$; let $M$ be an $A$-module and $N$ a finitely generated $A$-module, and let $u: M \rightarrow N$ be a homomorphism. If the induced homomorphism $M/\mathfrak{a}M \rightarrow N/\mathfrak{a}N$ is surjective, then $u$ is surjective.

\subsection{Solution}
TODO
\section{}
\subsection{Problem}
Let $A$ be a ring $\neq 0$. Show that $A^m \cong A^n \Rightarrow m = n$.

\subsection{Solution}
Let $\mathfrak{m}$ be a maximal ideal of $A$, let $k = A/\mathfrak{m}$, and let $\phi: A^m \rightarrow A^n$ be an isomorphism. Then $1 \otimes_k \phi: k \otimes_k A^m \rightarrow k \otimes_k A^n$ is an isomorphism of vector spaces of dimensions $m$ and $n$. Thus, $m = n$.

\section{}
\subsection{Problem}
Let $M$ be a finitely generated $A$-module and $\phi: M \rightarrow A^n$ a surjective homomorphism. Show that Ker $(\phi)$ is finitely generated.

\subsection{Solution}
Ker $(\phi) \subset M$ and $M$ is finitely generated, so Ker $(\phi)$ must be finitely generated.

\section{}
\subsection{Problem}
Let $f: A \rightarrow B$ be a ring homomorphism, and let $N$ be a $B$-module. Regarding $N$ as an $A$-module by restriction of scalars, form the $B$-module $N_B = B \otimes_A N$. Show that the homomorphism $g: N \rightarrow N_B$ which maps $y$ to $1 \otimes y$ is injective and that $g(N)$ is a direct summand of $N_B$.

\subsection{Solution}
TODO

\chapter{}

\section{}
\subsection{Problem}
Let $S$ be a multiplicatively closed subset of a ring $A$, and let $M$ be a finitely generated $A$-module. Prove that $S^{-1}M = 0$ if and only if there exists $s \in S$ such that $sM = 0$.

\subsection{Solution}
$\Leftarrow$: Let $m/t \in S^{-1}M$. Now note that $m/t  = 0/1 \Leftrightarrow mu = 0$ for some $u \in S$. But since $sM = 0$, $ms = 0$ for all $m \in M$, so setting $u = s$ establishes that $m/t = 0$, so $S^{-1}M = 0$.

$\Rightarrow$: Suppose that $S^{-1}M = 0$. Then for all $m/1 \in S^{-1}M$, we have $m/1 = 0/1$, so there exists a $u \in S$ such that $mu = 0$. Let $e_1, e_2, \cdots, e_n$ be the generators of $M$. Let their corresponding annihilators in $S$ be $u_1, u_2, \cdots, u_n$. Then $u_1 u_2 \cdots u_n$ annihilates every element of $M$. Thus, we are done.

\section{}
\subsection{Problem}
Let $\mathfrak{a}$ be an ideal of a ring $A$, and let $S = 1 + \mathfrak{a}$. Show that $S^{-1}\mathfrak{a}$ is contained in the Jacobson radical of $S^{-1}A$. 

\subsection{Solution}
I will show that $\mathfrak{a}$ is contained in every maximal ideal of $A$. Suppose not. Then there would exist $a \in \mathfrak{a}$, $m \in M$ such that $a + m = 1$. This implies that $a$ is a unit. Thus, $\mathfrak{a}$ is contained in every maximal ideal of $A$. It is also clear that $\mathfrak{a}$ and $S$ are disjoint. By the one-to-one correspondence between prime ideals of $A$ and prime ideals of $S^{-1}A$, this means that $S^{-1}\mathfrak{a}$ is contained in all the maximal ideals of $S^{-1}A$. The one-to-one correspondence applies to maximal ideals because they are disjoint from $S = 1 + \mathfrak{a}$. If $m = 1 + a$, this implies that $a$ is a unit. Therefore, $S^{-1}\mathfrak{a}$ is contained in the Jacobson radical of $S^{-1}A$, as desired.


\section{}
\subsection{Problem}
Let $A$ be a ring, let $S$ and $T$ be two multiplicatively closed subsets of $A$, and let $U$ be the image of $T$ in $S^{-1}A$. Show that the rings $(ST)^{-1}A$ and $U^{-1}(S^{-1}A)$ are isomorphic.

\subsection{Solution}
Let $f: (ST)^{-1}A \rightarrow U^{-1}(S^{-1}A)$ be defined by $f(a/st) = (a/s)/(t/1)$. This map is clearly surjective. Now suppose that $f(a_1/s_1t_1) = f(a_2/s_2t_2)$. Then $(a_1/s_1)/(t_1/1) = (a_2/s_2)/(t_2/1)$. Then there is some $a_3/s_3 \in S^{-1}A$ such that $((a_1/s_1)(t_2/1) - (a_2/s_2)(t_1/1))a_3/s_3 = 0 \Rightarrow (a_1 t_2/s_1 - a_2 t_1/s_2)a_3/s_3 = 0 \Rightarrow (a_1t_2s_2 - a_2 t_1s_1)/s_1 s_2 \cdot (a_3/s_3) = 0 \Rightarrow a_1t_2s_2a_3 - a_2 t_1 s_1 a_3 = 0 $ This means that $a_1/(s_1t_1) = a_2/(s_2t_2)$, so $f$ is injective as well. Thus, $f$ is an isomorphism, as desired.

\section{}
\subsection{Problem}
Let $f: A \rightarrow B$ be a homomorphism of rings and let $S$ be a multiplicatively closed subset of $A$. Let $T = f(S)$. Show that $S^{-1}B$ and $T^{-1}B$ are isomorphic as $S^{-1}A$-modules.

\subsection{Solution}
TODO
\section{}
\subsection{Problem}
Let $A$ be a ring. Suppose that, for each prime ideal $\mathfrak{p}$, the local ring $A_{\mathfrak{p}}$ has no nilpotent element $\neq 0$. Show that $A$ has no nilpotent element $\neq 0$. If each $A_{\mathfrak{p}}$ is an integral domain, is $A$ necessarily an integral domain?

\subsection{Solution}

Suppose $x \in A$ was nilpotent and $\neq 0$. It must be contained in every prime ideal, so let $\mathfrak{p}$ be some prime ideal of $A$. Let $y \in A - \mathfrak{p}$, which is clearly not nilpotent, and consider the element $x/y \in A_{\mathfrak{p}}$. Suppose that $x^n = 0$. Then $(x/y)^n = x^n/y^n = 0/y^n$, which is equal to $0/1$, the zero element of $A_{\mathfrak{p}}$. This violates the assumption that $A_{\mathfrak{p}}$ has no nonzero nilpotent element. Thus we are done.

\section{}
\subsection{Problem}
Let $A$ be a ring $\neq 0$ and let $\Sigma$ be the set of all multiplicatively closed subsets $S$ of $A$ such that $0 \notin S$. Show that $\Sigma$ has maximal elements, and that $S \in \Sigma$ is maximal if and only if $A -S$ is a minimal prime ideal of $A$.
\subsection{Solution}
$\Sigma$ has maximal elements by Zorn's lemma, because $A - \{0\}$ is a multiplicatively closed subset that contains all the others, and inclusion is a partial order. Suppose $S \in \Sigma$ is maximal. Consider $A - S$. If this was not prime, then $S$ would not be multiplicatively closed, because there would exist $x,y \in S$ such that $xy \in A - S$. Thus, $A-S$ is prime. Suppose that $A - S$ was not minimal, and suppose $T \subset S$ was a proper subset of $S$ forming a prime ideal. Then $A - T$ would be multiplicatively closed and contain $S$, which contradicts the fact that $S$ is maximal. Thus, $A-S$ is a minimal prime ideal, as desired.

For the other direction, suppose $A-S$ is a minimal prime ideal. Then clearly $S$ is multiplicatively closed by the definition of prime ideals. If $S$ was not maximal, then suppose $S \subset T$ for some $T$ that is multiplicatively closed. Then $A - T$ would be prime and be contained in $A - S$ which contradicts the fact that $A-S$ is minimal. Thus, we are done.

\section{}
\subsection{Problem}
A multiplicatively closed subset $S$ of a ring $A$ is said to be \emph{saturated} if \[xy \in S \Leftrightarrow x,y \in S.\]
Prove that 

i) $S$ is saturated $\Leftrightarrow$ $A - S$ is a union of prime ideals.

ii) If $S$ is any multiplicatively closed subset of $A$, there is a unique smallest saturated multiplicatively closed subset $\bar{S}$ containing $S$, and that $\bar{S}$ is the complement in $A$ of the union of prime ideals which do not meet $S$. ($\bar{S}$ is called the \emph{saturation} of $S$.)

If $S = 1 + \mathfrak{a}$, where $\mathfrak{a}$ is an ideal of $A$, find $\bar{S}$.

\subsection{Solution}
.

i) Suppose $A-S$ is a union of prime ideals. Let $x,y \in S$. Then $xy$ cannot be in any of the prime ideals comprising $A-S$, so $xy \in S$. Next, suppose $xy \in S$. WLOG, if $x \in A - S$, then $x$ is in some prime ideal in $A-S$, so $xy $ is also in the same prime ideal, implying $xy \in A - S$, a contradiction. Thus, $x,y \in S$. Thus, $S$ is saturated.

For the other direction, suppose $S$ is saturated. Let $T$ be the union of all prime ideals which do not meet $S$. Let $x \in (A - S) - T$, and let $y \in S$. If $xy \in S$, then $x \in S$, a contradiction. If $xy \in T$, then $xy$ is in some prime ideal, but both $x$ and $y$ are not in that prime ideal, a contradiction. Thus, $xy \in (A-S) - T$. One can clearly see that $(A-S) - T$ must be a prime ideal, and it does not meet $S$, a contradiction because we defined $T$ be the union of all prime ideals that do not meet $S$. Thus, $(A-S)-T$ must be the empty set, so $A-S = T$, so $A-S$ is the union of prime ideals, as desired.

ii) Let $T$ be the union of all prime ideals that do not meet $S$, and consider $A-T$. This contains $S$, and is saturated by part i. Suppose there was a smaller set $U$ containing $S$ that is saturated. Then let $x \in (A-T) - U$. Then applying the same argument from part i, we see that $(A-T) - U$ is a prime ideal which does not meet $S$, and hence must be in $T$, a contradiction. Thus, $T$ is in fact the unique smallest saturated subset containing $S$, as desired.

$\bar{1 + \mathfrak{a}}$ is the complement of the union of all prime ideals that do not meet $1 + \mathfrak{a}$.

\section{}
\subsection{Problem}
Let $S,T$ be multiplicatively closed subsets of $A$, such that $S \subset T$. Let $\phi: S^{-1}A \rightarrow T^{-1}A$ be the homomorphism which maps each $a/s \in S^{-1}A$ to $a/s$ considered as an element of $T^{-1}A$. Show that the following statements are equivalent:

i) $\phi$ is bijective.

ii) For each $t \in T, t/1$ is a unit in $S^{-1}A$.

iii) For each $t \in T$ there exists $x \in A$ such that $xt \in S$.

iv) $T$ is contained in the saturation of $S$ (Exercise 7)

v) Every prime ideal which meets $T$ also meets $S$.

\subsection{Solution}
.

i $\Rightarrow$ ii: Since $\phi$ is surjective, for each $a/t \in T^{-1}A$, there exists $a_1/s \in S^{-1}A$ such that $a_1/s$, considered as an element of $T^{-1}A$, is equal to $a/t$. So $\exists u \in A$ such that $u(ta_1 - sa) = 0$.

ii) TODO

iii) TODO

iv) TODO

v) TODO




\end{document}


