\documentclass[book,12pt,oneside,openany]{memoir}
\usepackage[utf8x]{inputenc}
\usepackage[english]{babel}
\usepackage{amsmath}
\usepackage{amssymb}
\usepackage{url}

% for placeholder text
\usepackage{lipsum}

\title{Solutions to \emph{Introduction to Commutative Algebra} by Atiyah and Macdonald}
\author{Aditya Gudibanda}

\begin{document}
\maketitle

\chapter{}

\section{}
\subsection{Problem}
Let $x$ be a nilpotent element of a ring $A$. Show that $1 + x$ is a unit of  $A$. Deduce that the sum of a nilpotent element and a unit is a unit.

\subsection{Solution}
Note the identity $\left(\sum_{i = 0}^k x_i \right) (1-x) =  1- x^{k+1}$. Since $x$ is nilpotent,  $x^l = 0$ for some $l \geq 1$. Thus, if we set $k = l-1$ in the identity above, we see that $1-x$ has a multiplicative inverse, and is therefore  a unit. 

To prove the original statement, given that $x$ is a nilpotent unit, it is clear that $-x$ is nilpotent as well, and by the logic above, $1 - (-x) = 1 + x$ is a unit, as desired.

Suppose $u$ is an arbitrary unit. Since $x$ is nilpotent, $u^{-1}x$ is nilpotent as well, so by the above, $u^{-1}x + 1$ is a unit. Since the product of units is a unit, $u(u^{-1}x + 1) = x + u$ is a unit, as desired.

\section{}
\subsection{Problem}
Let $A$ be a ring and let $A[x]$ be the ring of polynomials in an indeterminate $x$, with coefficients in $A$. Let $f = a_0 + a_1x + \cdots + a_n x^n \in A[x]$. Prove that

i) $f$ is a unit in $A[x] \Leftrightarrow a_0$ is a unit in $A$ and $a_1, \ldots, a_n$ are nilpotent. 

ii) $f$ is nilpotent $\Leftrightarrow a_0, a_1, \ldots, a_n$ are nilpotent.

iii) $f$ is a zero-divisor $\Leftrightarrow$ there exists $a \neq 0$ in $A$ such that $af = 0$.

iv) $f$ is said to be primitive if $(a_0, a_1, \ldots, a_n) = (1)$. Prove that if $f,g \in A[x]$, then $fg$ is primitive $\Leftrightarrow f$ and $g$ are primitive.

\subsection{Solution}

i) 


\section{}
\subsection{Problem}
Generalize the results of Exercise 2 to a polynomial ring $A[x_1, \ldots, x_r]$ in several indeterminates.

\subsection{Solution}

\section{}
\subsection{Problem}
In the ring $A[x]$, the Jacobson radical is equal to the nilradical.

\subsection{Solution}

\section{}
\subsection{Problem}
Let $A$ be a ring and let $A[[x]]$ be the ring of formal power series $f = \sum_{n=0}^{\infty} a_n x^n$ with coefficients in $A$. Show that 

i) $f$ is a unit in $A[[x]] \Leftrightarrow a_0$ is a unit in $A$.

ii) If $f$ is nilpotent, then $a_n$ is nilpotent for all $n \geq 0$. Is the converse true?

iii) $f$ belongs to the Jacobson radical of $A[[x]] \Leftrightarrow a_0$ belongs to the Jacobson radical of $A$.

iv) The contraction of a maximal ideal $\mathfrak{m}$ of $A[[x]]$ is a maximal ideal of $A$, and $\mathfrak{m}$ is generated by $\mathfrak{m}^c$ and $x$.

v) Every prime ideal of $A$ is the contraction of a prime ideal of $A[[x]]$.


\subsection{Solution}

\section{}
\subsection{Problem}
A ring $A$ is such that every ideal not contained in the nilradical contains a nonzero idempotent (that is, an element $e$ such that $e^2 = e \neq 0$). Prove that the nilradical and Jacobson radical of $A$ are equal.
\subsection{Solution}



\section{}
\subsection{Problem}
Let $A$ be a ring in which every element $x$ satisfies $x^n = x$ for some $n > 1$ (depending on $x$). Show that every prime ideal in $A$ is maximal.
\subsection{Solution}



\section{}
\subsection{Problem}
Let $A$ be a ring $\neq 0$. Show that the set of prime ideals of $A$ has minimal elements with respect to inclusion.
\subsection{Solution}



\section{}
\subsection{Problem}
Let $\mathfrak{a}$ be an ideal $\neq (1)$ in a ring $A$. Show that $\mathfrak{a} = r(\mathfrak{a}) \Leftrightarrow \mathfrak{a}$ is an intersection of prime ideals.
\subsection{Solution}



\section{}
\subsection{Problem}
Let $A$ be a ring, $\mathcal{R}$ its nilradical. Show that the following are equivalent:

i) $A$ has exactly one prime ideal

ii) every element of $A$ is either a unit or nilpotent

iii) $A/\mathcal{R}$ is a field
\subsection{Solution}



\section{}
\subsection{Problem}
A ring $A$ is Boolean if $x^2 = x$ for all $x \in A$. In a Boolean ring $A$, show that 

i) $2x = 0$ for all $x \in A$

ii) every prime ideal $\mathfrak{p}$ is maximal, and $A/\mathfrak{p}$ is a field with two elements

iii) every finitely generated ideal in $A$ is principal.
\subsection{Solution}



\section{}
\subsection{Problem}
A local ring contains no idempotent $\neq 0,1$.
\subsection{Solution}



\section{}
\subsection{Problem}
Let $K$ be a field and let $\Sigma$ be the set of all irreducible monic polynomials $f$ in one indeterminate with coefficients in $K$. Let $A$ be the polynomial ring over $K$ generated by indeterminates $x_f$, one for each $f \in \Sigma$. Let $\mathfrak{a}$ be the ideal of $A$ generated by the polynomials $f(x_f)$ for all $f \in \Sigma$. Show that $\mathfrak{a} \neq (1)$.

Let $\mathfrak{m}$ be the maximal ideal of $A$ containing $\mathfrak{a}$, and let $K_1 = A/\mathfrak{m}$. Then $K_1$ is an extension field of $K$ in which each $f \in \Sigma$ has a root. Repeat the construction with $K_1$ in place of K, obtaining a field $K_2$, and so on. Let $L = \cup_{n=1}^{\infty} K_n$. Then $L$ is a field in which each $f \in \Sigma$ splits completely into linear factors. Let $\bar{K}$ be the set of all elements of $L$ which are algebraic over $K$. Then $\bar{K}$ is an algebraic closure of $K$.
\subsection{Solution}



\section{}
\subsection{Problem}
In a ring $A$, let $\Sigma$ be the set of all ideals in which every element is a zero-divisor. Show that the set $\Sigma$ has maximal elements and that every maximal element of $\Sigma$ is a prime ideal. Hence the set of zero-divisors of $A$ is a union of prime ideals.
\subsection{Solution}



\section{}
\subsection{Problem}
Let $A$ be a ring and let $X$ be the set of all prime ideals of $A$. For each subset $E$ of $A$, let $V(E)$ denote the set of all prime ideals of $A$ which contain $E$. Prove that

i) if $\mathfrak{a}$ is the ideal generated by $E$, then $V(E) = V(\mathfrak{a}) = V(r(\mathfrak{a}))$.

ii) $V(0) = X, V(1) = \emptyset$.

iii) If $(E_i)_{i \in I}$ is any family of subsets of $A$, then \[V (\cup_{i \in I} E_i ) = \cap_{i \in I} V(E_i).\]

iv) $V(\mathfrak{a} \cap \mathfrak{b}) = V(\mathfrak{a}\mathfrak{b}) = V(\mathfrak{a}) \cup V(\mathfrak{b})$ for any ideals $\mathfrak{a}, \mathfrak{b}$ of $A$.

\subsection{Solution}



\section{}
\subsection{Problem}
Draw pictures of Spec $\mathbb{Z}$, Spec $\mathbb{R}$, Spec $\mathbb{C}[x]$, Spec $\mathbb{R}[x]$, Spec $\mathbb{Z}[x]$.
\subsection{Solution}



\section{}
\subsection{Problem}

\subsection{Solution}



\section{}
\subsection{Problem}

\subsection{Solution}



\section{}
\subsection{Problem}
A topological space $X$ is said to be irreducible if $X \neq \emptyset$ and if every pair of non-empty open sets in $X$ intersect, or equivalently if every non-empty open set is dense in $X$. Show that Spec $A$ is irreducible if and only if the nilradical of $A$ is a prime ideal.
\subsection{Solution}



\section{}
\subsection{Problem}

\subsection{Solution}










\end{document}